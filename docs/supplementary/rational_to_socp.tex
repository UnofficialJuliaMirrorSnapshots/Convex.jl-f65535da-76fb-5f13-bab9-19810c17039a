\documentclass[11pt]{article}

% \usepackage{statistics-macros}
\usepackage{hyperref}
\usepackage[numbers,square]{natbib}
\usepackage[top=2.54cm,bottom=2.54cm,left=2.54cm,right=2.54cm]{geometry}
\usepackage{color}
\usepackage{enumerate}
\usepackage{amssymb}
\usepackage{amsbsy}
\usepackage{amsmath}
\usepackage{amsthm}
\usepackage{amsfonts}
\usepackage{latexsym}
\usepackage{graphicx}
\usepackage{xifthen}
\usepackage{xspace}
\usepackage{bbm}

\newcommand{\R}{\mathbb{R}}
\newcommand{\N}{\mathbb{N}}
\newcommand{\half}{\frac{1}{2}}
\newcommand{\ceil}[1]{\left\lceil{#1}\right\rceil}
\newcommand{\norm}[1]{\left\|{#1}\right\|} % A norm with 1 argument
\newcommand{\lone}[1]{\norm{#1}_1} % l1 norm
\newcommand{\ltwo}[1]{\norm{#1}_2} % l2 norm
\newcommand{\linf}[1]{\norm{#1}_\infty} % l-infinity norm

\begin{document}

\begin{center}
  {\LARGE $p$-norm cones and second-order cone representations} \\
  \vspace{.2cm}
  {\large John C.\ Duchi} \\
\end{center}

In this note, we review how a few power inequalities, and related $p$-norm
constraints for (rational) $p \in (1, \infty)$, may be represented as second
order cone programs. These results follow~\citet{AlizadehGo01}, who
enumerate the reductions below and several more such inequalities.

Our starting point is to note that given a vector $w \in \R^n$ and points
$x, y \ge 0$, the constraint
\begin{equation*}
  \ltwo{w} \le xy
\end{equation*}
is equivalent to the constraint
\begin{equation}
  \label{eqn:rotated-cone}
  \ltwo{\left[\begin{matrix} 2w \\ x - y \end{matrix}\right]}
  \le x + y,
  ~~ x \ge 0, y \ge 0,
\end{equation}
which is evident by squaring both sides and rearranging.  Thus, we also
obtain that the set of $t, x, y$ such that $x, y \ge 0$ and $t^2 \le xy$ is
representable as a second-order cone. This insight allows us to represent
more complicated powers as second-order cones.

\subsection*{Products as second order cones}

Consider the set of $t \in \R, s \in \R^{2^k}_+$ such that
\begin{equation}
  \label{eqn:two-k-set}
  t^{2^k} \le s_1 s_2 \cdots s_{2^k},
  ~~ s_i \ge 0, ~ \mbox{all}~ s_i.
\end{equation}
We claim this set is SOCP representable. Indeed,
introduce variables
\begin{equation*}
  \left\{u_{i,j} :
  j \in \{1, \ldots, k-1\},
  i \in \{1, \ldots, 2^j\},
  u_{i,j} \ge 0\right\}.
\end{equation*}
Then it is clear that the constraint~\eqref{eqn:two-k-set}
is equivalent to
\begin{equation*}
  t^{2^k} \le u_{1, k-1}^2 u_{2,k-1}^2 \cdots u_{2^{k-1}, k-1}^2,
  ~~ u_{i, k-1}^2 \le s_{2i - 1} s_{2i},
  ~~ \mbox{all~} i,
\end{equation*}
or
\begin{equation*}
  t^{2^{k-1}} \le u_{1, k-1} u_{2,k-1} \cdots u_{2^{k-1}, k-1},
  ~~ u_{i, k-1}^2 \le s_{2i - 1} s_{2i},
  ~~ \mbox{all~} i \in \{1, \ldots, 2^{k-1}\}.
\end{equation*}
Recursively applying this construction through levels $j = k-2, \ldots,
1$, we obtain the set of inequalities
\begin{align}
  t^2 \le u_{1,1} u_{2,1},
  & ~~ u_{i,j-1}^2 \le u_{2i-1, j} u_{2i,j}
  ~ \mbox{for ~} j \in \{2, \ldots, k-2\}, i \in \{1, \ldots, 2^{j-1}\}
  \nonumber \\
  & ~~ u_{i, k-1}^2 \le s_{2i - 1} s_{2i}
  ~~ \mbox{for~} i \in \{1, \ldots, 2^{k-1}\}, \label{eqn:socp-two-k-set}
\end{align}
where all $u_{i,j}$ are non-negative. Each of these inequalities, by the
representation~\eqref{eqn:rotated-cone}, corresponds to a second-order
cone in $\R_+^3$, and we have introduced $2^k - 1$ such inequalities.

% Suppose we have the convex constraint $x^p \le t$, where

\subsection*{Product inequalities as second order cones}

We now provide reductions of inequalities of the more restrictive form
\begin{equation}
  \label{eqn:initial-power}
  x^n \le t^{p_1} s^{p_2}, ~~ x, t, s \ge 0
\end{equation}
where $p_1 + p_2 = n$ and $p_1, p_2, n \in \N$, into a sequence of
second-order cone-represented sets. This is the building block for
representations of general $p$-norm inequalities to come. In particular, we
develop a procedure that takes inequalities of one of the three forms
\begin{subequations}
  \label{eqn:power-recursions}
  \begin{align}
    \label{eqn:zero-power}
    x^n & \le t^{p_1} s^{p_2} \\
    \label{eqn:one-power}
    x^n & \le t^{p_1} s^{p_2} u \\
    \label{eqn:two-power}
    x^n & \le t^{p_1} s^{p_2} u^2
  \end{align}
\end{subequations}
and recurses with a power $\le (n + 1)/2$ on $x$ and one of the
forms~\eqref{eqn:power-recursions}. Note that if $n = 2$, then we already
have an immediate second order cone representation~\eqref{eqn:rotated-cone}.

\begin{enumerate}[(1)]
\item The case with $u^0$, inequality~\eqref{eqn:zero-power}.
  We have two possibilities: either $n$ is even or $n$ is odd.
  \begin{enumerate}[a.]
  \item $n$ is even: we assume that $n \ge 4$, as otherwise we have
    the inequality $x^2 \le ts$, which is trivial.
    In this case, either both $p_1$ and $p_2$ are even or
    both are is odd.  If they are even, inequality~\eqref{eqn:zero-power} is
    equivalent to $x^{n/2} \le t^{p_1/2} s^{p_2 / 2}$, which gives us a
    recursive step. If $p_1$ is odd, then $p_2$ is odd, and
    inequality~\eqref{eqn:zero-power} is equivalent to the pair
    \begin{equation*}
      x^n \le t^{p_1 - 1} s^{p_2 - 1} u^2, ~~ u^2 \le ts,
      ~~~ \mbox{or} ~~~
      x^{n/2} \le t^{\frac{p_1 - 1}{2}} s^{\frac{p_2 - 1}{2}} u,
      ~~ u^2 \le ts,
    \end{equation*}
    which allows us to recurse to case~\eqref{eqn:one-power} with lower
    powers on $s$, $t$, and $x$.
  \item $n$ is odd: In this case, we have exactly one of $p_1$ and $p_2$ is
    odd; assume w.l.o.g.\ that $p_1$ is odd. Then
    inequality~\eqref{eqn:zero-power} is equivalent to
    $x^{n + 1} \le t^{p_1 - 1} s^{p_2} x t$, and introducing the variable
    $u \ge 0$, we have the equivalent representation
    \begin{equation*}
      x^{n + 1} \le t^{p_1 - 1} s^{p_2} u^2, ~~ u^2 \le xt,
      ~~~ \mbox{or} ~~~
      x^{\frac{n + 1}{2}} \le t^{\frac{p_1 - 1}{2}} s^{\frac{p_2}{2}} u,
      ~~ u^2 \le xt.
    \end{equation*}
    This allows us to recurse to case~\eqref{eqn:one-power}.
  \end{enumerate}
\item The case $u$, inequality~\eqref{eqn:one-power}. We again have
  two possibilities: either $n$ is even or $n$ is odd.
  \begin{enumerate}[a.]
  \item $n$ is even: In this case, exactly one of $p_1$ and $p_2$ is odd;
    assume w.l.o.g.\ that $p_1$ is odd. Then we have the equivalent
    representation
    \begin{equation*}
      x^n \le t^{p_1 - 1} s^{p_2} w^2, ~~ w^2 \le tu,
      ~~~ \mbox{or} ~~~
      x^{\frac{n}{2}} \le t^{\frac{p_1 - 1}{2}} s^{\frac{p_2}{2}} w,
      ~~ w^2 \le tu.
    \end{equation*}
    This is again an inequality of the form~\eqref{eqn:one-power}.
  \item $n$ is odd: In this case, either both $p_1$ and $p_2$ are even
    or they are both odd. If they are both even, then we have the equivalent
    inequalities
    \begin{equation*}
      x^{n + 1} \le t^{p_1} s^{p_2} w^2, ~~ w^2 \le xu,
      ~~~ \mbox{or} ~~~
      x^{\frac{n + 1}{2}} \le t^{\frac{p_1}{2}} s^{\frac{p_2}{2}} w,
      ~~ w^2 \le xu,
    \end{equation*}
    which is again of the form~\eqref{eqn:one-power}. If both $p_1$ and $p_2$
    are odd, then we introduce a few more variables,
    noting that inequality~\eqref{eqn:one-power} is equivalent to
    \begin{equation*}
      x^{n + 1} \le t^{p_1 - 1} s^{p_2 - 1} st ux,
      ~~~ \mbox{or} ~~~
      x^{n + 1} \le t^{p_1 - 1} s^{p_2 - 1} y^4,
      ~~ y^2 \le w v, ~~ w^2 \le st, ~~ v^2 \le ux,
    \end{equation*}
    and raising the inequality involving $x^{n + 1}$ to the power $\half$
    yields the four inequalities
    \begin{equation*}
      x^{\frac{n + 1}{2}} \le t^{\frac{p_1 - 1}{2}} s^{\frac{p_2 - 1}{2}} y^2,
      ~~ y^2 \le w v, ~~ w^2 \le st, ~~ v^2 \le ux,
    \end{equation*}
    which is of the form~\eqref{eqn:two-power}.
  \end{enumerate}
\item The case $u^2$, inequality~\eqref{eqn:two-power}. We show that
  we can always reduce this to one of the cases~\eqref{eqn:power-recursions}.
  \begin{enumerate}[a.]
  \item $n$ is even: In this case, either both $p_1$ and $p_2$ are both even
    or they are both odd. If $p_1$ and $p_2$ are even, then
    inequality~\eqref{eqn:two-power} is equivalent to $x^{n/2} \le
    t^{p_1/2} s^{p_2/2} u$, which is of type~\eqref{eqn:one-power}.
    If $p_1$ and $p_2$ are odd, then we have the equivalent representation
    \begin{equation*}
      x^n \le t^{p_1 - 1} s^{p_2 - 1} u^2 st, ~~~ \mbox{or} ~~~
      x^n \le t^{p_1 - 1} s^{p_2 - 1} y^4, ~~ y^4 \le u^2 w^2,
      ~~ w^2 \le st,
    \end{equation*}
    which (by inspection) is equivalent to the inequality of
    type~\eqref{eqn:two-power} (along with an additional two SOCP inequalities)
    \begin{equation*}
      x^\frac{n}{2} \le t^\frac{p_1 - 1}{2} s^\frac{p_2 - 1}{2} y^2,
      ~~ y^2 \le u w,
      ~~ w^2 \le st.
    \end{equation*}
  \item $n$ is odd: In this case, exactly one of $p_1$ and $p_2$ is odd;
    assume w.l.o.g.\ that $p_1$ is odd. Then inequality~\eqref{eqn:two-power}
    is equivalent to
    \begin{equation*}
      x^{n + 1} \le t^{p_1 - 1} s^{p_2} u^2 t x,
      ~~~ \mbox{or} ~~~
      x^\frac{n + 1}{2} \le t^\frac{p_1 - 1}{2} s^\frac{p_2}{2}
      y^2,
      ~~ y^2 \le uw, ~~ w^2 \le tx.
    \end{equation*}
  \end{enumerate}
\end{enumerate}

The preceding enumerated steps suggest a recursive strategy, where at each
step, we check which of the cases we are in, then perform a reduction to
introduce at most three inequalities of the form $w^2 \le uv$, and reducing
the powers on all other variables by a factor of $2$. The recursion halts as
soon as we have an inequality of the form $x^n \le y^n$, or an inequality of
the form $x^2 \le uv$, which is second-order-cone representable.
Note that given an inequality of the form~\eqref{eqn:initial-power},
we introduce at most $O(1) \ceil{\log_2( p_1 \vee p_2)}$ new
inequalities via this recursion.

\subsection*{General (rational) $p$-norm cones as second-order cones}

Now we show how to represent the inequality
\begin{equation}
  \label{eqn:p-norm-cone}
  \norm{x}_p \le t,
  ~~~ \mbox{where~} p = \frac{n}{m}
  ~ \mbox{for~some~} n, m \in \N, n \ge m
\end{equation}
for $x \in \R^d$
as a collection of second-order cones and linear inequalities.
First, note that the inequality
\begin{equation*}
  \bigg(\sum_{i=1}^d |x_i|^p \bigg)^{1/p} \le t
  ~~ \equiv ~~
  \bigg(\sum_{i=1}^d |x_i|^{n/m} \bigg)^{m/n} \le t
  ~~ \equiv ~~
  \sum_{i=1}^d t^{1 - n/m} |x_i|^{n/m} \le t
\end{equation*}
is equivalent to the collection of inequalities
$s^\top 1 \le t$, $|x_i|^{n/m} \le s_i t^{n/m - 1}$ for all $i$,
which in turn is equivalent to the set of inequalities
\begin{equation*}
  v_i \ge |x_i|, ~ t \ge 0, ~ s_i \ge 0, ~ v_i^n \le s_i^m t^{n - m},
  ~ \sum_{i=1}^d s_i \le t.
\end{equation*}
We know this is SOCP representable by the
recursions~\eqref{eqn:power-recursions}.

\bibliographystyle{abbrvnat}
\bibliography{bib}

\end{document}
